\titledquestion{Multiple Choices}

Each question has \textbf{one or more}  answer(s). Select all the  answer(s). For each question, you will get 0 points if you select one or more wrong answers, but you will get 1 point if you select a non-empty subset of the  answers.

Write your answers in the following table.

%%%%%%%%%%%%%%%%%%%%%%%%%%%%%%%%%%%%%%%%%%%%%%%%%%%%%%%%%%%%%%%%%%%%%%%%%%%
% Note: The `LaTeX' way to answer a multiple-choices question is to replace `\choice'
% with `\Choice', as what you did in the previous questions. However, there are 
% still many students who would like to handwrite their homework. To make TA's work 
% easier, you have to fill your selected choices in the table below, no matter whether 
% you use LaTeX or not.
%%%%%%%%%%%%%%%%%%%%%%%%%%%%%%%%%%%%%%%%%%%%%%%%%%%%%%%%%%%%%%%%%%%%%%%%%%%

\begin{table}[htbp]
	\centering
	\begin{tabular}{|p{2cm}|p{2cm}|p{2cm}|p{2cm}|}
		\hline
		(a) & (b) & (c) & (d) \\
		\hline
		%%%%%%%%%%%%%%%%%%%%%%%%%%%%%%%%%%%%%%%%%%%%%%%%%%%%%%%%%%
		% YOUR ANSWER HERE.
		  AC  &  C  & C   & ABCD  \\
		%%%%%%%%%%%%%%%%%%%%%%%%%%%%%%%%%%%%%%%%%%%%%%%%%%%%%%%%%%
		\hline
	\end{tabular}
\end{table}

\begin{parts}
	\part[2] Which of the followings are true?  

	\begin{choices}
	    \CorrectChoice One Floyd-Warshall algorithm's step is to find $d^{(k)}_{i,j}$: that is, the shortest path allowing intermediate visits to vertices $v_1,v_2, ...,v_{k-1},v_k$.
		\choice Like Dijkstra's algorithm, Floyd-Warshall algorithm cannot work with any graph with negative weight edge.
        \CorrectChoice Floyd-Warshall algorithm can find the shortest path between all pairs of nodes while Dijkstra's algorithm is used to find single-source shortest path.
		\choice Floyd-Warshall algorithm uses greedy idea.
	\end{choices}
	
	\part[2] Which of the followings are true?  

	\begin{choices}
		\choice For A* graph search algorithm, it will always return an optimal solution if it exists.
		\choice For A* graph search algorithm with admissible heuristic, it will always return an optimal solution if it exists.
		\CorrectChoice For A* graph search algorithm with consistent heuristic, it will always return an optimal solution if it exists.
		\choice None of the above.
	\end{choices}
	
	\part[2] We run Floyd-Warshall algorithm on a graph with $n$ vertices $v_1,v_2,...,v_{n/2},..., v_n$ ($n$ is even). Suppose all three loops $(k, i, j)$ are iterated from $1$ to $n$. After running at least ______ iterations of the out-most loop $k$, it is ensured to find the shortest path between $v_{n/2}$ and $v_n$. 
		
	\begin{choices}
	    \choice $\frac{n}{2}-1$
	    \choice $\frac{n}{2}$
		\CorrectChoice $n-1$
		\choice $n$
	\end{choices}
	
	\part[2] Which of the following statements about shortest path algorithms is/are true?
	\begin{choices} 
	    \CorrectChoice If we modify the outer loop of Dijkstra’s algorithm to execute $|V|-1$ iterations instead of $|V|$ iterations (i.e. only pop $|V|-1$ times from the heap), the algorithm can still find the shortest path on a positive-weighted graph.
		\CorrectChoice We can modify Bellman-Ford algorithm to detect whether there exists a negative cycle or not in a directed graph.
		\CorrectChoice If we modify the outer loop of Bellman-Ford algorithm to execute $|V|$ iterations instead of $|V|-1$ iterations (i.e. apply update rule to each edge for $|V|$ times), the algorithm can still find the shortest path on a positive-weighted graph.
		\CorrectChoice We can modify Floyd-Warshall algorithm to detect whether there exists a negative cycle or not in a directed graph.
	\end{choices}
	
\end{parts}