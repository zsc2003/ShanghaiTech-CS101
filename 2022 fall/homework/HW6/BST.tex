\titledquestion{BST and AVL tree}

\begin{parts}
    \part[3] Draw a valid BST of minimum height containing the keys 1, 2, 4, 6, 7, 9, 10.\\\\\\\\
    \begin{tikzpicture}
    \Tree
    [.6
        \edge[]; [.2
                \edge[]; [.1 ]
                \edge[]; [.4 ]
            ]
        \edge[]; [.9
                \edge[]; [.7 ]
                \edge[]; [.10 ]
            ]
    ]
    \end{tikzpicture}
    \vspace{1cm}
    
    \part[4] Given an empty AVL tree, insert the sequence of integers $15, 20, 23, 10, 13, 7, 30, 25$ from left to right into the AVL tree. Draw the final AVL tree.\\\\\\\\
    
    \begin{tikzpicture}
    \Tree
    [.13
        \edge[]; [.10
                \edge[]; [.7 ]
                \edge[blank]; \node[blank]{};
            ]
        \edge[]; [.20
                \edge[]; [.15 ]
                \edge[]; [.25
                    \edge[]; [.23 ]
                    \edge[]; [.30 ]
                ]
            ]
    ]
    \end{tikzpicture}
    \vspace{1cm}
   
    \part[2]For the final AVL tree in the question (b), delete $7$. Draw the AVL tree after deletion.
    \begin{tikzpicture}
    \Tree
    [.20
        \edge[]; [.13
                \edge[]; [.10 ]
                \edge[]; [.15 ]
            ]
        \edge[]; [.25
                \edge[]; [.23 ]
                \edge[]; [.30 ]
            ]
    ]
    \end{tikzpicture}
    \vspace{1cm}
   
   \newpage
   
   \part[5]For an AVL tree, define D = the number of descendants of the left child of the root - the number of descendants of the right child of the root. Then what is the maximum of D for an AVL tree with height n? 
   
   $D_{max}=k_1\times 2^n+k_2\times B^n+k_3\times(-\frac{1}{B})^n$, please write down the value of B and $k_i$.
   
   since we want to make $D_{max}$\\
   so we can make the number of descendants of the left child of the root as big as possible,\\
   and make the number of descendants of the right child of the root as small as possible.\\
   for the left child, the max situation is that the left subtree is a perfect tree with the height of $n-1$,
   and since its height is $n-1$,so its number of nodes is $2^n-1$.\\
   and for right subtree, the min number of nodes to mention the tree as a AVL tree,\\
   so the min height of the right subtree is $n-2$, and the number is min nodes with height of $n-2$ is  
   $F(n-2)=\frac{1}{\sqrt{5}}[(\frac{\sqrt{5}+1}{2})^{n+1}-(\frac{1-\sqrt{5}}{2})^{n+1}]-1$\\
   so $D_{max}=$left\_max\_number-right\_min\_number$=2^n-1-\{\frac{1}{\sqrt{5}}[(\frac{\sqrt{5}+1}{2})^{n+1}-(\frac{1-\sqrt{5}}{2})^{n+1}]-1\}$\\
   $=2^n-\frac{1}{\sqrt{5}}(\frac{\sqrt{5}+1}{2})^{n+1}+\frac{1}{\sqrt{5}}(-\frac{1}{\frac{\sqrt{5}+1}{2}})^{n+1}$\\
   $=2^n-\frac{1+\sqrt{5}}{2\sqrt{5}}(\frac{\sqrt{5}+1}{2})^n+\frac{1-\sqrt{5}}{2\sqrt{5}}(-\frac{1}{\frac{\sqrt{5}+1}{2}})^n$\\
   $=k_1\times 2^n+k_2\times B^n+k_3\times(-\frac{1}{B})^n$\\\\
   so above all\\
   $k_1=1$\\
   $k_2=-\frac{1+\sqrt{5}}{2\sqrt{5}}$\\
   $k_3=\frac{1-\sqrt{5}}{2\sqrt{5}}$\\
   $B=\frac{\sqrt{5}+1}{2}$
   

\end{parts}