\titledquestion{Similarity Test}

Maddelena has drawn a lot of paintings. She wants to organize her work. To finish her job better, she decides to determine whether a subset of similar paintings exists. Every paint $p$ has its own feature $\lambda_p$. 

In this problem, we specify some notations: $\mathcal{P}$ is the set of paintings. Without giving rise to ambiguity, $\lambda_p$ can be referred to the feature of a painting $p$. $\mathcal{F} = \{\lambda_p|p\in \mathcal{P}\}$ denotes the set of features. $n$ can be refereed to the number of paintings i.e. $n=\left|\mathcal{P}\right|$.

Note that $\mathcal{F}$ is not partially ordered i.e. you can't make comparisons for any $p,q\in \mathcal{P}$ so that you cannot sort $\mathcal{F}$ or give out inferences like $\lambda_p<\lambda_q$ or $\lambda_q<\lambda_p$. There exists an equivalence relation on $\mathcal{F}$, as $\lambda_p = \lambda_q$ indicates $p$ and $q$ has similar features.

The master feature is defined as the feature it is similar to \textbf{over} half of the paintings. 
In other words, the "master" feature is the feature$\lambda_0\in \mathcal{F}$ s.t. $|\{p\in\mathcal{P}|\lambda_p = \lambda_0\}|>\frac n2$.

For example, $2$ sets of paintings are shown: The first set of paintings has a "master" feature $\lambda_0 = \alpha$ since there exists over a half ($4$: a,b,e,g) paintings share this feature. The second set of paintings does not have a ``master" feature because there exists no such feature that over half paintings share that.
\begin{table}[!ht]
    \centering
    \begin{minipage}[b]{.5\linewidth}
    \centering
    \begin{tabular}{|l|l|l|l|l|l|l|l|}
    \hline
        Painting & a & b & c & d & e & f & g \\ \hline
        Feature & $\alpha$ & $\alpha$ & $\beta$ & $\omega$ & $\alpha$ & $\beta$ & $\alpha$ \\ \hline
    \end{tabular}
    \caption{A "master" feature $\lambda_0 = \alpha$.}
    \end{minipage}
    \begin{minipage}[b]{.4\linewidth}
    \centering
    \begin{tabular}{|l|l|l|l|l|l|}
    \hline
        Painting & A & B & C & D & E \\ \hline
        Feature & $\alpha$ & $\beta$ & $\omega$ & $\beta$ & $\alpha$ \\ \hline
    \end{tabular}
    \caption{No "master" feature.}
    \end{minipage}
\end{table}
\newpage
\begin{parts}
\part{} 
Maddelena wants to find a "master" feature first. She asks you to help her with those problems:
\begin{subparts}
    \subpart[3] {Design a divide-and-conquer algorithm whose worst case takes $\Theta(n\log n)$ time.} \\
    \textbf{Note: Of course, you can index them, but we recommend you \red{not} to.} \\
    \textbf{Hint:} Recall merge sort, think out the relation between the "master" features of the $2$ subsets after dividing and the "master" feature of the original set. (If they have)
   \begin{solution} 
    \\
    \\
    \\
    \\
    \\
    \\
    \\
    \\ 
    \\
    \\
    \\
    \\
    \\
    \\
    \\
    \\
    \\
    \\
    \\
    \\
    \end{solution}
    \subpart[2] {Justify it by proving its correctness and show its time complexity is $\Theta(n\log n)$. }
    \begin{solution} 
    \\
    \\
    \\
    \\
    \\
    \\
    \\
    \\
    \\
    \end{solution}
\end{subparts}

\part{} 
Maddelena finds $\Theta(n\log n)$ still not fast enough since she has so many paintings. She wants to develop a new algorithm that takes $o(n\log n)$ time.  \\
\textbf{In this problem, you can always assume that the total remaining number is even. In other words, you don't need to consider how to deal with the one superfluous after dividing.}

\textbf{Hint: Recall binary search, what if we give up at least half every recurrence? \\ Match those by pairs and do differently based on whether they are similar.}

\begin{subparts}
    \subpart[2] {Design a divide-and-conquer algorithm that takes $o(n\log n)$ time.}
    \begin{solution}
    \\
    \\
    \\
    \\
    \\
    \\
    \\
    \\
    \end{solution}
    \subpart[2] {Justify its correctness.}
    \begin{solution}
    \\
    \\
    \\
    \\
    \\
    \\
    \\
    \\ 
    \\
    \\
    \\
    \\
    \\
    \end{solution}
    \subpart[1] {Analyze the time complexity of the algorithm.}
    \begin{solution}
    \\
    \\
    \end{solution}
\end{subparts}

\end{parts}