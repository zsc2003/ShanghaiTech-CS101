\titledquestion{Multiple Choices}

Each question has \textbf{one or more} correct answer(s). Select all the correct answer(s). For each question, you will get 0 points if you select one or more wrong answers, but you will get 1 point if you select a non-empty subset of the correct answers.

Write your answers in the following table.

%%%%%%%%%%%%%%%%%%%%%%%%%%%%%%%%%%%%%%%%%%%%%%%%%%%%%%%%%%%%%%%%%%%%%%%%%%%
% Note: The `LaTeX' way to answer a multiple-choices question is to replace `\choice'
% with `\CorrectChoice', as what you did in the first question. However, there are still
% many students who would like to handwrite their homework. To make TA's work easier,
% you have to fill your selected choices in the table below, no matter whether you use 
% LaTeX or not.
%%%%%%%%%%%%%%%%%%%%%%%%%%%%%%%%%%%%%%%%%%%%%%%%%%%%%%%%%%%%%%%%%%%%%%%%%%%

\begin{table}[htbp]
	\centering
	\begin{tabular}{|p{2cm}|p{2cm}|p{2cm}|p{2cm}|}
		\hline 
		(a) & (b) & (c) & (d) \\
		\hline
		%%%%%%%%%%%%%%%%%%%%%%%%%%%%%%%%%%%%%%%%%%%%%%%%%%%%%%%%%%
		% YOUR ANSWER HERE.
		C & AD & AC & D \\
		%%%%%%%%%%%%%%%%%%%%%%%%%%%%%%%%%%%%%%%%%%%%%%%%%%%%%%%%%%
		\hline
	\end{tabular} 
\end{table}

\begin{parts}
	\part[2] Assume we insert one element before the element at index \(i(0\leqslant i\leqslant n-1)\) in an array of length \(n\) and get a new array of length \(n+1\). What is the minimum number of moves that the elements in the array need in total?

	\begin{choices}
		\choice \(0\)
		\choice \(i\)
		\CorrectChoice \(n-i\)
		\choice \(n-i-1\)
	\end{choices}

	\part[2] Which of the following statements about arrays and linked-lists are true?

	\begin{choices}
		\CorrectChoice Gaining access to the \(k\)-th element in an array takes constant time.
		\choice Gaining access to the \(k\)-th element in a linked-list takes constant time.
		\choice Erasing the \(k\)-th element in an array takes constant time.
		\CorrectChoice With access to the \(k\)-th element, inserting an element after the \(k\)-th element in a linked-list takes constant time.
	\end{choices}

	\part[2] Please evaluate the following reverse-Polish expressions using stacks. Which of the following expressions are \textbf{illegal}?

	\begin{choices}
		\CorrectChoice \ttt{1 2 * + 3 5 +}
		\choice \ttt{4 5 6 / * 1 /}
		\CorrectChoice \ttt{1 + 2 - 3 + 4}
		\choice \ttt{7 8 9 1 + - *}
	\end{choices}

	\part[2] Assume we implement a queue with a circular array indexed from \(0\) to \(n-1\), and the \ttt{front} pointer is currently at index \(m\). We will know that the queue is full if the \ttt{back} pointer is at index \fillin[][0.5in]. (Options below are mapped to integers modulo \(n\): \(\mathbb{Z}_n=\{0,1,2,\cdots,n-1\}\).)

	\begin{choices}
		\choice \(0\)
		\choice \(m\)
		\choice \(n-1\)
		\CorrectChoice \(m-1\)
	\end{choices}
\end{parts}