\titledquestion{Let's Play with Linked-List, Stack and Queue}

In the following questions, we want to implement the \ttt{push} and \ttt{pop} operations of stacks and queues with singly-linked-lists in the most efficient way. You can only access the head node of the linked-list. Access to the tail node is not allowed.

You can answer this question using pseudocode or natural language. Please make sure your description is clear and easy to understand.

\begin{parts}
	\part[5]\label{part:5a} \textbf{Stack Using Linked-List}\par
	Please specify how you implement stack's \ttt{push} and \ttt{pop} operations.
	\begin{solution}
		%%%%%%%%%%%%%%%%%%%%%%%%%%%%%%%%%%%%%%%%%%%%%%%%%%%%%%%%%%%%%
		% Replace the `vspace{2.5in}' with your answer.
		%%%%%%%%%%%%%%%%%%%%%%%%%%%%%%%%%%%%%%%%%%%%%%%%%%%%%%%%%%%%%
		%\vspace{2.5in}

		To implement the \ttt{push} and \ttt{pop} operations of stacks by using Linked-List
		
		we need to set a structure called Node, each node contains the some elements that we need to store,
		and a next pointer, points to the node which is the early one to enter the stack 
		
		we also need a head pointer, points to the top node of the stack

		the implement of push operator:
		
		\begin{cpp}
void push(Node* new_element)
{
  Node* new_head = new_element;
  new_head->next = head;
  head = new_element;
}
		\end{cpp}

		for pop operator, one thing needs to notice is that if the stack is empty, we can not execute the pop instruction
		
		so we need to check whether the stack is empty or not before removing the top node of the stack
		
		the implement of pop operator:
		
		\begin{cpp}
int pop()
{
  if(head->next == nullptr)
    return 0;   // the stack is empty, 
			// we failed to execute the pop instruction
  Node* old_head = head;
  head = head->next;
  delete old_head;
  old_head = nullptr;
  return 1;       // the stack is not empty, 
			// we successed to execute the pop instruction
}
		\end{cpp}

	\end{solution}

	\newpage
	
	\part[5] \textbf{Queue Using Linked-Lists}\par
	Please specify how you implement queue's \ttt{push} and \ttt{pop} operations with linked-list operations. You may directly use the operations you implemented in part (\ref{part:5a}).\par
	\emph{\textbf{Hint:} Use multiple stacks to implement a queue.}
	\begin{solution}
		%%%%%%%%%%%%%%%%%%%%%%%%%%%%%%%%%%%%%%%%%%%%%%%%%%%%%%%%%%%%%
		% Replace the `vspace{2.5in}' with your answer.
		%%%%%%%%%%%%%%%%%%%%%%%%%%%%%%%%%%%%%%%%%%%%%%%%%%%%%%%%%%%%%
		%\vspace{2.5in}

		To implement the \ttt{push} and \ttt{pop} operations of queues by using Linked-List
		
		we can use two stacks to implement a queue
		name two stacks A and B (each has their own head pointer: headA ,head B)
		
		for the push implement:

		1. we need to check whether stack B is empty
		   
			if stack B is not empty, we need to remove all the elements in stack B into stack A with the push and pop implments of stacks
		
		2. after stack B becomes empty, we put the new element into stack A

		the implement of push operator:
		
		some part of the function is pseudocode(stackA. means the operations of stack A , stackB. means the operations of stack B)
		
		\begin{cpp}
void push(Node* new_element)
{
  if(headB->next != nullptr)
  {
    while(headB->next != nullptr)
    {
      stackA.push(stackB.top());
      stackB.pop();
    }
  }
  stackA.push(new_element);
}
		\end{cpp}

		for the pop implement:
		
		1. we need to check whether stack A is empty
		   
			if stack A is not empty, we need to remove all the elements in stack A into stack B with the push and pop implments of stacks
		
		
		2. after stack A becomes empty, we need to check whether stack B is empty
			
			a. if stack B is empty, then it means that the queue is empty, so we fail to implement pop operate this time
			
			b. if stack B is not empty, we just need to pop the top element of stack B

		the top() operate means that get the head element of the stack
		
		the implement of pop operator:
		
		some part of the function is pseudocode(stackA. means the operations of stack A , stackB. means the operations of stack B)
		
		\begin{cpp}
int pop()
{
  if(headA->next != nullptr)  // the stackA is not empty, 
  {		    // we need to remove the elements into stackB
    while((headA->next != nullptr)
    {
      stackB.push(stackA.top());
      stackA.pop();
    }
  }
  if(headB->next == nullptr)  // the stackB is empty,
    return 0;   // we failed to execute the pop instruction
  stackB.pop();
  return 1;       // the stackB is not empty, 
	        // we successed to execute the pop instruction
}
		\end{cpp}
	\end{solution}
\end{parts}