\titledquestion{Discovery}

\begin{parts}
  \part[2] Is C++ STL \lstinline{std::sort} stable or not? If not, is there any stable sort function provided by the standard library?
  \begin{solution}
    %%%%%%%%%%%%%%%%%%%%%%%%%%%%%%%%%%%%%%%%%%%%%%%%%
    % Replace `\vspace{0.5in}' with your answer.
    %\vspace{0.5in}
    \(C++ \ STL \ std::sort\) is not stable.

    but the \(std::stable\_sort\) is stable.

    %%%%%%%%%%%%%%%%%%%%%%%%%%%%%%%%%%%%%%%%%%%%%%%%%
  \end{solution}
  \part[0] Suppose \(A\) is an array of size \(n\). If we can find the median value of \(A\) within \(O(n)\) time, it is possible to make quick-sort \(\Theta(n\log n)\) in worst case. STFW (Search The Friendly Web) about how to find the median value in \(O(n)\) time.
  
  \begin{solution}
    %%%%%%%%%%%%%%%%%%%%%%%%%%%%%%%%%%%%%%%%%%%%%%%
    
    In the book \(Introduction oto algorithms\), and by searching the friendly internet, I found an algorithm
    called \(BFPRT\), the name is from five of its inventors.

    the algorithm is mainly decided by the following steps.
    
    \begin{itemize}
      \item divide the array into groups every five adjacent numbers
      \item for each group of numbers, we find the median of the five numbers and form a median array of the median of all groups
      \item the median of each group forms a new array
      \item recursively do the above steps
      \item finally there reamin only one number, the medium number of the whole array.
      
    \end{itemize}
    
    \(T(n) \leq  T(\frac{n}{5}) +T(\frac{7n}{10})+c \cdot n\)

    and it is proved on book and internet that \(T(n) \leq 10c \cdot n\)

    so \(T(n)=\Theta(n)\)
    
    actually, when I was learning algorithms before, I hearded that we can use stl in C++, the \(std::nth\_element(...)\) can find the 
    \(k\_th\) minimal element with in \(\Theta(n)\) time.

    the median element can be easily find by using it, so I just want to write it before scearching

    However, when I scearch the code of it clearly, I thought, and some analysis on the internet also 
    say that the time complexity of \(std::nth_element\) is \(\Theta(n)\) on average, instead of the worst case,
    the worst case might turn into \(\Theta(n^2)\)

    so it might not be the best way to find the medium number, the \(DFPRT\) algorithm can get the median in \(\Theta(n)\)
    %%%%%%%%%%%%%%%%%%%%%%%%%%%%%%%%%%%%%%%%%%%%%%%
  \end{solution}
  
  \part[0] It is known that some sorting algorithms, like quick-sort, need to swap elements. Run the following code, change the value of \(n\) and see how the output changes.
  \begin{cpp}
    #include <algorithm>
    #include <cstdlib>
    #include <iostream>
    #include <vector>

    namespace std {
      template <>
      inline void swap<int>(int &lhs, int &rhs) noexcept {
        auto tmp = lhs;
        lhs = rhs;
        rhs = tmp;
        std::cout << "swap is called.\n";
      }
    } // namespace std

    int main() {
      std::srand(19260817);
      constexpr int n = 10;
      std::vector<int> vec;
      for (auto i = 0; i != n; ++i)
        vec.push_back(std::rand());
      std::sort(vec.begin(), vec.end());
      return 0;
    }
  \end{cpp}
  From your observation, the \lstinline{swap} function is never called when \(n\leqslant\) \fillin[][2em]. What algorithm(s) does \lstinline{std::sort} use?
  \begin{solution}
    %%%%%%%%%%%%%%%%%%%%%%%%%%%%%%%%%%%%%%%%%%%%%%%
    % Replace `\vspace{0.7in}' with your answer.
    %\vspace{0.7in}
    the \lstinline{swap} function is never called when \(n\leqslant 16\) .

    the std::sort is based on the introspective sorting. It is combined with many different type
    of sorting algorithms, and it will execute different onces if it reach some different situations.

    the threshold of the std::sort is 16.

    \begin{itemize}
      \item the length of the array need to be sorted is less than or equal to the threshold
      the std::sort will execute the insertion-sort.
      
      \item the length of the array need to be sorted is more than the threshold,
        
      there is another value \(depth\_limit\), which usually set as \(2*logN\), N is the length of the array need to be sorted

      \begin{itemize}
        \item the time of the recursion is less than or equal to the \(depth\_limit\)
        
        it will execute the recursive quick-sort.

        \item the time of the recursion more than the \(depth \_ limit\)
        
        it will execute heap-sort.
      \end{itemize}

    \end{itemize}
     

    %%%%%%%%%%%%%%%%%%%%%%%%%%%%%%%%%%%%%%%%%%%%%%%
  \end{solution}
\end{parts}