\titledquestion{Multiple Choices}

Each question has \textbf{one or more} correct answer(s). Select all the correct answer(s). For each question, you will get 0 points if you select one or more wrong answers, but you will get 1 point if you select a non-empty subset of the correct answers.

Write your answers in the following table.

%%%%%%%%%%%%%%%%%%%%%%%%%%%%%%%%%%%%%%%%%%%%%%%%%%%%%%%%%%%%%%%%%%%%%%%%%%%
% Note: The `LaTeX' way to answer a multiple-choices question is to replace `\choice'
% with `\CorrectChoice', as what you did in the previous questions. However, there are 
% still many students who would like to handwrite their homework. To make TA's work 
% easier, you have to fill your selected choices in the table below, no matter whether 
% you use LaTeX or not.
%%%%%%%%%%%%%%%%%%%%%%%%%%%%%%%%%%%%%%%%%%%%%%%%%%%%%%%%%%%%%%%%%%%%%%%%%%%

\begin{table}[htbp]
	\centering
	\begin{tabular}{|p{2cm}|p{2cm}|p{2cm}|p{2cm}|}
		\hline
		(a) & (b) & (c) & (d) \\
		\hline
		%%%%%%%%%%%%%%%%%%%%%%%%%%%%%%%%%%%%%%%%%%%%%%%%%%%%%%%%%%
		% YOUR ANSWER HERE.
		  BC  &  B   &  B  & C  \\
		%%%%%%%%%%%%%%%%%%%%%%%%%%%%%%%%%%%%%%%%%%%%%%%%%%%%%%%%%%
		\hline
	\end{tabular}
\end{table}

\begin{parts}
	\part[2] Which of the following is/are true about Trees?

	\begin{choices}
		\choice A full binary tree with n nodes has height of $O(\ln n)$.
		\CorrectChoice Only given the depth of all leaf nodes in a tree, we can infer the height of this tree.
		\CorrectChoice If $T$ was transformed from a general tree $T'$ through \textbf{Knuth transform}, which means $T$ is a left-child right-sibling binary tree. Then the post-order traversal of $T'$ identical to the in-order traversal of $T$.
		\choice None of the above.
	\end{choices}
	
	\part[2] Let’s look at some magic properties about trees, which of the following statements is/are true?

	\begin{choices}
		\choice A tree is a full binary tree if and only if every node has an odd number of descendants.
		\CorrectChoice A rooted binary tree has the property that the number of leaf nodes equals to the number of full nodes plus $1$.
		\choice There are 133 distinct shapes of binary trees with 6 nodes.
		\choice None of the above.
	\end{choices}
	
	\part[2] Which of the following statements about the binary heap is/are true? Note that binary heaps mentioned in this problem are implemented by complete binary trees. 
		
	\begin{choices}
	    \choice If a binary tree is a min-heap, then the post-order traversal of this tree is a descending sequence.
	    \CorrectChoice We have a binary heap of $n$ elements and wish to add n more elements into it while maintaining the heap property. It can be done in $O(n)$.
		\choice There exists a heap with seven distinct elements so that the in-order traversal gives the element in sorted order. 
		\choice If item A is an ancestor of item B in a heap, then it must be the case that the $Push$ operation for item A occurred before the $Push$ operation for item B.
		\choice None of the above.
	\end{choices}
	
	\part[2] Which traversals of the left tree and right tree, will produce the same sequence node name? \\
\begin{minipage}{1\textwidth}
\centering
\begin{tikzpicture}
\Tree
[.A
    [.B
        \edge[blank]; \node[blank]{};
        \edge[];[.D
            \edge[];[.E
                \edge[];[.G ]
                \edge[];[.H
                    \edge[blank]; \node[blank]{};
                    \edge[];[.I 
                        \edge[blank]; \node[blank]{};
                        \edge[];[.J
                        ] 
                    ]
                ]
            ]
            \edge[]; [.F
            ]
        ]
    ]
    [.C
    ]
]
\end{tikzpicture}
\begin{tikzpicture}
\Tree
[.G
    \edge[blank]; \node[blank]{};
    \edge[]; [.F
        \edge[]; [.E 
            \edge[];[.I 
                \edge[];[.J ]
                \edge[];[.H ]
            ]
            \edge[blank]; \node[blank]{};
        ]
        \edge[]; [.C 
            \edge[]; [.D
                \edge[blank]; \node[blank]{};
                \edge[];[.B ]  
            ]
            \edge[]; [.A ]
        ] 
    ]
]
\end{tikzpicture}
\end{minipage} \\

	\begin{choices} 
	    \choice left: Post-order, right: Post-order 
		\choice left: Pre-order, right: Pre-order 
		\CorrectChoice left: Post-order, right: In-order
		\choice left: In-order, right: In-order 
	\end{choices}
	
\end{parts}